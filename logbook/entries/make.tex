Projects consist of tasks that depend on each other.
If an upstream task changes, downstream tasks ought to be re-run using that new input.
This is easy to do by hand when your project has a small number of tasks.
This is hard to do when the tasks are computationally intensive and the input-output graph is complicated.
Thankfully, computer programmers addressed these difficulties long ago with a Unix utility called \href{https://en.wikipedia.org/wiki/Make_(software)}{Make}.

Makefiles are machine-readable documentation that make your workflow reproducible.
They follow naturally from our task-based approach.
Downstream tasks depend on upstream tasks.
Using \texttt{make} requires you to specify those input-output relationship (dependencies).
When generated files are missing, or when files they depend on have changed, needed files are re-made using a sequence of commands you specify.
The commands are language-agnostic: if you can type it at the shell prompt, 
\texttt{make} can execute it.

\subsection{Learn Make}

I won't reinvent the wheel by explaining further.
Here are four introductions to \texttt{make}, listed in the order that I suggest reading them:
\begin{itemize}
\item Mike Bostock: \href{https://bost.ocks.org/mike/make/}{Why Use Make}
\item Karl Broman: \href{http://kbroman.org/minimal_make/}{minimal make}
\item Kieran Healy: \href{http://plain-text.co/pull-it-together.html}{Pull It Together} (The Plain Person's Guide to Plain Text Social Science)
\item Zachary M. Jones: \href{http://zmjones.com/make/}{GNU Make for Reproducible Data Analysis}
\end{itemize}

\subsection{Notes on writing Makefiles}

The \texttt{Makefile} in the \texttt{logbook} folder is a simple example.
It compiles this PDF if \texttt{logbook.tex} or one of the elements of \texttt{\$(logbookentries)} is newer than \texttt{logbook.pdf}.

An important reminder: Each line in the recipe must start with a tab.
See \href{https://www.gnu.org/software/make/manual/html_node/Recipe-Syntax.html}{Recipe Syntax}.

A few notes on Makefile style.
\begin{itemize}
\item It is helpful to define \href{https://www.gnu.org/software/make/manual/html_node/Variables-Simplify.html#Variables-Simplify}{variables} containing long lists of files, such as \texttt{inputs= file1 file2 file3}
so that you can summarize dependencies simply by writing \texttt{\$(inputs)}
and define targets simply by writing \texttt{all: \$(inputs) \$(outputs)}.
However, it is only sensible to write a target-dependency relationship as
\texttt{\$(outputs): \$(inputs)}
if there is one recipe that produces all those outputs jointly.
\item Writing separate recipes for separate targets is advantageous because Make will return more informative errors by telling you which recipe failed.
For example, instead of having a script \texttt{create\_symlinks.sh} that tries to produce the symbolic links for every input file,
you can write a recipe for every symbolic link in which the symbolic link in the task's \texttt{input} folder is a target.
When one of those symbolic links fails to be created, Make will tell you which symbolic link didn't work, making it easier to identify the problematic/missing input upstream.
\end{itemize}

A note on running Makefiles in a cluster computing environment that has a job scheduler:
\begin{itemize}
	\item A problem with batch jobs is that the shell sees the submission command as completed upon job submission, before any output files are produced.
	Failing to see the output, Make will repeatedly submit the job.
	\item You need the job submission command to not exit back to the shell until the job completes, so that the target will be produced before Make evaluates that recipe's success or failure.
	\item With \texttt{sbatch} at Chicago Midway, use \texttt{sbatch -W}.
	\item With \texttt{qsub} at Census RDC, use \texttt{qsub -W block=true}.
	\item Read \url{http://wresch.github.io/2012/11/01/qsub-submit-jobs-from-makefile.html} for more discussion of \texttt{qsub} and the \texttt{sbatch} \href{https://slurm.schedmd.com/sbatch.html}{manual} for discussion of \texttt{-W} or \texttt{--wait}. 
\end{itemize}

Pages of the \texttt{Make} manual I find myself often consulting:
\begin{itemize}
	\item \href{https://www.gnu.org/software/make/manual/html_node/Prerequisite-Types.html}{4.3 Types of Prerequisites}
	\item \href{https://www.gnu.org/software/make/manual/html_node/Multiple-Targets.html}{4.10 Multiple Targets in a Rule}
	\item \href{https://www.gnu.org/software/make/manual/html_node/Static-Usage.html}{4.12.1 Syntax of Static Pattern Rules}
	\item \href{https://www.gnu.org/software/make/manual/html_node/Automatic-Variables.html}{10.5.3 Automatic Variables}
\end{itemize}

O'Reilly's \href{https://www.oreilly.com/openbook/make3/book/ch12.pdf}{``Debugging Makfiles'' chapter} is helpful.